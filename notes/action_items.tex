\documentclass[12pt, preprint]{aastex}
\usepackage{bm}

\input{vc}
\newcommand{\sectionname}{Section}
\newcommand{\documentname}{\textsl{Note}}

\newcommand{\set}[1]{\bm{#1}}
\newcommand{\mean}[1]{\overline{#1}}
\newcommand{\given}{\,|\,}
\newcommand{\teff}{\mbox{$\rm T_{eff}$}}
\newcommand{\feh}{\mbox{$\rm [Fe/H]$}}
\newcommand{\logg}{\mbox{$\rm \log g$}}
\usepackage{morefloats}
\begin{document}


\section{ACTION ITEMS}

for Monday 8am:
Plots for DWH: 
1 x training data on the isochrones: make the legend the same colour as the points
1 x all stars run through on the isochrone: '' '' 

for week starting Oct 1 
- generalise code N labels
- make s using optimiser rather than min of points 
- test not only the lower SNR but also test the fewer pixels 
- can add in alpha/Fe and see what this is (but need to email Jon Holtzman to try to get this in a fits file format) 

HWR
1. for self-fits plots Make not output v input but output-input v input
2. test - do we just need clusters at the extreemes
3. See marie martig
4. 



DWH\\
 
1. Need to generalise the code for N labels\\
2. Change the github repository from APOGEE to The Cannon\\
3. Make individual pixel plots of the Teff, logg, [Fe/H] as a function of the flux - the other parameters so we are removing the other terms - details in notebook\\
4. Do more of these plots for the linear case above\\
4. the covariance should be coming out of the optimisers - done\\
5. Change in the optimiser absolute$\_$sigma = True: not possible $/$ not an option with my version of the code\\
6. Do cross validation test on take one cluster out, take one star out, including with age as a label\\

HWR \\
1. Need to introduce priors so as can convolve Likelihood (what we return) with the priors\\
2. Need to plot teff-log g in density plane, demonstrate that the red clump is there.\\
3. Make AMR plot for all but with calibration data on this figure\\
4. add covariances into the error estimation - 3/4 dimensional gaussian sum mapping onto each axis - find the maths for this and apply to the covariance matrix.\\
5. Need to plot where the binaries are on the Teff - log g plot - this is simple there is a flag for this.\\
6. Need to understand why the stars are in imaginary parameter space - checked SNR it is not the SNR.\\
7. There are 10 x as many giants as dwarfs. If x giants by a 0.1 weighting, does the rgb get worse but the main sequence remain the same - suspect this is so.\\

JB \\
1. Compare the RC and lower RGB to ASPCAP results to see if The Cannon gets this right.\\

MKN \\
1. Need to clean up the code  - currently have two versions fitspectra.py and fitspectra$\_$ages.py - adding more dimensions of inforamtion to these.
- also currently make a cut of Tdiff < 600 to remove Pleiades stars with high reddening/dispersion from estimated temp. from ASPCAP - brittle way of going about things and has to be carried through
when plotting self (input-output calibration stars) figures.\\
2. Make Teff -log g plot of all with density\\
3. Run through bulge fields with latest version (different normalistation)\\
Take care of the normalisation - get 0.1 dex offset in [Fe/H] when use not flux continuum normalised APOGEE spectra as initial input (different normalisation procedure to training labels)\\
4.Look at the spectra that is returned versus the best match - these look more different than I would imagine\\
5. I want to take the training set of data off the ASPCAP scale: Use temperatures from the IR flux method, use the same [Fe/H] for all clusters, use log g from the isochrones\\
6. Should I remove outliers in the self 1:1 plot?\\
7. We need to add rotation in as a label (dwarfs) . Note ASPCAP DR12 dwarf results are not reliable. Note there is apparently offsets (hotter) temperatures by 100K for DR12 compared to DR10 and better\\
agreement of APOGEE in matching the isochrones (I think our agreement is already good - look at this). JH says metallicity distribution of the field 4201 is slightly different in DR12 compared to DR10\\
8. There are hotter dwarf calibrators (Diane and Jon working on these) : would be good to get these for calibration of hotter stars : these are in DR12. Also emailed Jennifer Johnson about this.\\

CBJ
Should not just use metadata - mean but use (metadata - mean)/std as this will put them on a comparable scale - had discussed this in the summer with HWR and DWH wrt making the plot of the\\
coefficients\\
\\
AH
Could use Kepler sample of stars with ages to make a training set to return ages for , for red clump only. This is a really good subproject.\\

PAPER1 \\
- Add priors so that stars fall on the isochrones\\
- Express that this is simplest case\\
- HWR will change nonclamenture to intutitive\\
- Characterise a number for each pixel that describes how it changes with each parameter, that is normalised so on the same scale; these could be the coefficients, but HW suggested positive assertion:\\
0 - no dependence , +ve , dependence, in this way characterise which regions have strongest dependence - would lose iformation about +ve or +ve slope though which is interesting. Why do some lines\\
show negative slope with [Fe/H] and others positive for example? Is there any relation to other dimensions of information as a function of these ratios of slopes - age/chem tagging applications?\\

paper 2 \\
Katia Cunha - working with her to get individual abundance measurements and alpha elementes
Ages - this is a very good region to explore. It will be possible to get age information in this way from the spectra.

MM \\
Has new age plots should look at these

MF \\
Has suggestions wrt the code and how to imporve this - should implement these (also download JEDI + plugins)

DONE \\
1. DIBS test done - taken out had no effect in the plane on the params\\
but - if I take weaker lines for cool stars (continuum cut) get a better agreement with APOGEE results for field 4330 - need to examine this more closely - not sure which is correct\\
2. ADDED AGE AS A LABEL - there is age information in the spectra\\
3. Added chi2 result fit for all stars\\
4. Fixed covs that now comes out of optimiser but absolute$\_$sigma = True not implentable in my version of python\\

\section{TEST PLOTS} 

\subsection{Leave one cluster out cross validation}

\begin{figure}
  \includegraphics[width=\hsize]{./plots/cluster_legend.png}
  \end{figure}
\begin{figure}
  \includegraphics[width=\hsize]{./plots/Pleiades_out.png}
  \end{figure}
\begin{figure}[h!]
  \includegraphics[width=\hsize]{./plots/N7789_out.png}
  \end{figure}
\begin{figure}[h!]
  \includegraphics[width=\hsize]{./plots/N6819_out.png}
  \end{figure}
\begin{figure}[h!]
  \includegraphics[width=\hsize]{./plots/N6791_out.png}
  \end{figure}
\begin{figure}[h!]
  \includegraphics[width=\hsize]{./plots/N5466_out.png}
  \end{figure}
\begin{figure}[h!]
  \includegraphics[width=\hsize]{./plots/N4147_out.png}
  \end{figure}
\begin{figure}[h!]
  \includegraphics[width=\hsize]{./plots/N2420_out.png}
  \end{figure}
\begin{figure}[h!]
  \includegraphics[width=\hsize]{./plots/N2158_out.png}
  \end{figure}
\begin{figure}[h!]
  \includegraphics[width=\hsize]{./plots/N188_out.png}
  \end{figure}
\begin{figure}[h!]
  \includegraphics[width=\hsize]{./plots/M92_out.png}
  \end{figure}
\begin{figure}[h!]
  \includegraphics[width=\hsize]{./plots/N5466_out.png}
    \end{figure}
\begin{figure}[h!]
  \includegraphics[width=\hsize]{./plots/M71_out.png}
    \end{figure}
    
\begin{figure}[h!]
  \includegraphics[width=\hsize]{./plots/M67_out.png}
    \end{figure}

\begin{figure}[h!]
  \includegraphics[width=\hsize]{./plots/M53_out.png}
    \end{figure}
\begin{figure}[h!]

  \includegraphics[width=\hsize]{./plots/M5_out.png}
    \end{figure}
\begin{figure}
  \includegraphics[width=\hsize]{./plots/M3_out.png}  
  \end{figure}
\begin{figure}

  \includegraphics[width=\hsize]{./plots/M2_out.png}
    \end{figure}
\begin{figure}

  \includegraphics[width=\hsize]{./plots/M15_out.png}
    \end{figure}
\begin{figure}

  \includegraphics[scale=0.5]{./plots/M13_out.png}
     \end{figure}
  \begin{figure}
  \includegraphics[scale=0.5]{./plots/M107_out.png}
   \end{figure}




\end{document}
