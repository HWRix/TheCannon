% This file is part of the Apogee project.
% Copyright 2014 Melissa Ness and David W. Hogg.

\documentclass[12pt, preprint]{aastex}

\begin{document}

Hello world

\section{Introduction}

Challenges of stellar parameter estimation.

In principle there could be data-driven models.

Data-driven models would permit ``label transfer''.
This is especially important if we want optical models to provide labels for infrared surveys (and so on).

APOGEE provides a great data set for exploring these issues.

\section{Data}

read in data; which files, etc.

re-continuum-normalize; how?

demonstrate that the normalization is good.

meta-data on cluster stars; why do we believe it; table

\section{Spectral model}

linear model plus scatter.

Subtract off and save mean values of meta-data.

likelihood function.

implementation and outputs, figures.

is this a good model?

\section{Parameter estimation}

we want to label other stars

index concept

spectral fitting concept

methodological details

leave-one-out cross-validation on the training spectra

leave-one-cluster-out cross-validation

application to some example APOGEE targets

comparison to ASPCAP

\section{Discussion}

Did it work and etc?

Limitations of the training sample.

Linear model is a very rigid assumption; obviously wrong.
Generalize to a GP at every wavelength.

Applications, esp chemical tagging

\end{document}
